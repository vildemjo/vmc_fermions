\subsection{Two electrons in two dimensions}

We start with the simple case of two electrons in a harmonic oscillator trap. These electrons do not interact with each other and the trial wavefunction is given by Eq. \ref{eq:trial_wf_not_interacing}. 

\subsubsection{Brute force sampling}

First, brute force sampling was used to calculate the new position and evaluate the metropolis ratio. The double derivative of the wavefunction, used to calculate the kinetic energy part of the expectation energy, was evaluated both analytically and numerically. Table \ref{tab:brute_force_no_interaction_2p} shows the energy for different values for the parameter $\alpha$. The numbers show that the standard error of the mean (SEM) is underestimating the deviations. From Tab. \ref{tab:brute_force_no_interaction_2p} one can observe that including the correlations, i.e. mainly correlations between one state and the next where only one particle is moved, increases the deviation, giving us $\sigma_b$. This value is also an estimate of the error, but probably a more true estimate of the error.

From Tab. \ref{tab:brute_force_no_interaction_2p} one can also observe that $\alpha $ = 1.0 gives zero standard deviation and is therefore the optimal parameter. By comparing the results for the analytical and the numerical cases one can observe that the SEM and $\sigma_b$ is similar for both cases, especially around the ground state ($\alpha$ = 1.0). If he expectation energies from the analytical case and the numerical cases are compared, they differ with values at the scale of $10^{-3}$, which is reasonable with a $\sigma_b$ around $10^{-3}$ to $10^{-2}$. At last one can observe, both from the individual CPU time measurements and the mean CPU time of these 10 measurements (though with different $\alpha$s), that the analytical solution of the double derivative is much faster and more efficient than the numerical case. 

\begin{table}[H]\caption{Comparing the results for analytical/numerical evaluation of the double derivative. Here $\left< E_L \right>$ is the expectation value for the energy given in atomic units (a.u.) and CPU time is in units of seconds. $\sigma_B$ is the standard deviation after resampling with the blocking method and SEM is the standard deviation of the mean. Number of MC cycles are 2$^{21}$.}\label{tab:brute_force_no_interaction_2p}
\center
\begin{tabular}{l|l}
Analytical: &  Numerical:\\ \hline
\begin{tabular}{ccccc}
$\alpha$: & $\left< E_L \right>$: & SEM: & $\sigma_B$: & CPU time:\\ \hline
0.50 & 2.49402 & 0.00073 & 0.01022 & 5.57812\\
0.60 & 2.26441 & 0.00052 & 0.00690 & 5.76562\\
0.70 & 2.13118 & 0.00035 & 0.00448 & 5.92188\\
0.80 & 2.05016 & 0.00022 & 0.00263 & 5.67188\\
0.90 & 2.01015 & 0.00010 & 0.00116 & 5.96875\\
1.00 & 2.00000 & 0.00000 & 0.00000 & 5.62500\\
1.10 & 2.00871 & 0.00009 & 0.00102 & 6.20312\\
1.20 & 2.03402 & 0.00018 & 0.00175 & 6.34375\\
1.30 & 2.07259 & 0.00026 & 0.00244 & 5.95312\\
1.40 & 2.11041 & 0.00034 & 0.00311 & 6.15625\\ \hline
\end{tabular} & \begin{tabular}{ccccc}
$\alpha$: & $\left< E_L \right>$: & SEM: & $\sigma_B$: & CPU time:\\ \hline
0.50 & 2.49991 & 0.00073 & 0.01093 & 18.20310\\
0.60 & 2.26412 & 0.00053 & 0.00727 & 18.21880\\
0.70 & 2.13039 & 0.00036 & 0.00436 & 18.62500\\
0.80 & 2.04993 & 0.00022 & 0.00269 & 18.46880\\
0.90 & 2.01160 & 0.00010 & 0.00118 & 18.46880\\
1.00 & 2.00000 & 0.00000 & 0.00000 & 18.31250\\
1.10 & 2.00825 & 0.00009 & 0.00097 & 18.35940\\
1.20 & 2.03308 & 0.00018 & 0.00170 & 18.31250\\
1.30 & 2.06460 & 0.00026 & 0.00243 & 20.23440\\
1.40 & 2.11803 & 0.00033 & 0.00308 & 19.00000\\ \hline
\end{tabular}\\
Mean CPU time: 5.91875 & Mean CPU time:  18.62033\\
\end{tabular}
\end{table}

\subsubsection{Including importance sampling}

Figure \ref{fig:comparing_sampling} compare the expectation value of the energy and the acceptance rate of brute force sampling and importance sampling. It can be observed from the right part of the figure that the acceptance rate of both methods increase with decreasing step size, but one can also observe that the acceptance is lower for importance sampling than brute force sampling at large step sizes. These observations could indicate that a small step size would be ideal for both methods. 

\begin{figure}[H]
\center
\includegraphics[width=0.85\linewidth]{../Results/comparing_sampling}\caption{Left: Expectation energies after $2^{21}$ MC cycles for different step sizes. Right: Percentage of accepted steps for different step sizes. Here importance sampling and brute force sampling is compared. }\label{fig:comparing_sampling}
\end{figure}


From the left part of the figure it can be observed that one of the expectation values for the energies are lower than the ground state energy ($dl = 0.005$ with brute force sampling) when these calculations where done with $\alpha$ = 0.9. However, in Tab. \ref{tab:compare_stepsize_no_interaction_2p} which compare the result of brute force sampling and importance sampling for different step sizes one can observe that the standard deviation from the blocking method is larger for the case of brute force sampling with a step size of 0.005. However,  the SEM does not indicate anything to be special about this result.

\begin{table}[H]\caption{Comparing the results for importance/brute force sampling. Here the parameter $\alpha$ is set to 0.9 and number of MC cycles are 2$^{21}$. Acc. is short for acceptance and is here given in \% and $t_{CPU}$ is the CPU time used by the program in units of seconds. The rest of the values are as described in Tab. \ref{tab:brute_force_no_interaction_2p}. }\label{tab:compare_stepsize_no_interaction_2p}
\begin{tabular}{l|l|l} 
 & Brute force: &  Importance:\\ \hline
\begin{tabular}{c} 
$dl$:\\ \hline
1.000\\
0.500\\
0.100\\
0.050\\
0.010\\
0.005\\
0.001\\
\end{tabular} & \begin{tabular}{ccccc}
 $\left< E_L \right>$: & SEM: & $\sigma_B$: & Acc.: & $t_{CPU}$:\\ \hline
2.010 & 0.00010 & 0.00066 & 79.832& 6.625\\
2.011 & 0.00010 & 0.00112 & 89.794& 7.266\\
2.015 & 0.00011 & 0.00549 & 97.919& 7.078\\
2.007 & 0.00011 & 0.00946 & 99.001& 7.016\\
2.002 & 0.00007 & 0.01510 & 99.788& 6.953\\
1.942 & 0.00006 & 0.02140 & 99.916& 6.578\\
2.118 & 0.00002 & 0.00371 & 99.974& 6.531\\ \hline
\end{tabular} & \begin{tabular}{ccccc}
$\left< E_L \right>$: & SEM: & $\sigma_B$: & Acc.: & $t_{CPU}$:\\ \hline
2.011 & 0.00010 & 0.00022 & 71.078& 8.531\\
2.011 & 0.00010 & 0.00023 & 89.343& 8.672\\
2.011 & 0.00010 & 0.00047 & 99.049& 8.859\\
2.011 & 0.00010 & 0.00068 & 99.662& 8.188\\
2.007 & 0.00010 & 0.00136 & 99.968& 8.281\\
2.010 & 0.00010 & 0.00199 & 99.989& 8.547\\
2.007 & 0.00010 & 0.00410 & 99.999& 8.016\\ \hline
\end{tabular}\\
& Mean CPU time: 6.86384 & Mean CPU time: 8.44197 \\
\end{tabular}
\end{table}

I took a closer look at the actual local energies for the brute force sampling method. Figure \ref{fig:local_energy_step_size_brute_force} shows how the energy is not stable for steps sizes smaller than 0.01, so even though the step sizes 0.001 and 0.01 seems to give reasonable expectation values for the energy (see Tab. \ref{tab:compare_stepsize_no_interaction_2p} and Fig. \ref{fig:comparing_sampling}), Fig. \ref{fig:local_energy_step_size_brute_force} seems to show that that is sort of a lucky shot. I also saw this by running the calculation with brute force sampling and the step size, $0.005$, with different seeds for the random number generator. The expectation energy for five different runs where $\left< E_L \right>$ =  1.91487, 2.03452, 1.90805, 1.88356 and 1.9284. From Fig. \ref{fig:local_energy_step_size_brute_force} one can observe that even $dl=0.1$ seems to be too small since it also results in the local energy varying slowly and taking longer "trips" to higher energies and using many steps to get back down again, but for this step size the "trips" to higher energies are more frequent than for the smaller step sizes. I concluded that a step size of 0.5 is the best choice for the brute force sampling because it gives reasonable changes of the local energy and an acceptance rate of $\sim$ 90 \% (see Fig. \ref{fig:comparing_sampling}). 

\begin{figure}[H]
\center
\includegraphics[width=0.85\linewidth]{../Results/local_energy_step_sizes}\caption{The local energy for every tenth MC cycle for brute force sampling at different step sizes, $dl$. a) shows the smaller step sizes and b) some that are a bit larger.}\label{fig:local_energy_step_size_brute_force}
\end{figure}

%On the other hand, for the importance sampling, the investigation of the local energies did not give any clues to why the expecation energy is lower that the ground state energy for a step size of 0.5. Eventually I decided that the result itself indicate that 0.5 is a too large step size for importance sampling and taken together with a low acceptance rate at 0.5 ($\sim$ 82 \%) it is clear that a smaller step size is preferable. 
%
%At last, it is interesting to note that the CPU time of importance sampling is higher than the brute force sampling, so even though brute force sampling has a lower acceptance rate, at the ideal step size, than importance sampling the two methods seems to be somewhat comparable at least for the system I am investigating.

Proceeding to evaluate the energy, Table \ref{tab:ground_state_energy_different_omegas_no_interaction_2p} shows how the energy changes with different trap frequencies, $\omega$. From Tab. \ref{tab:ground_state_energy_different_omegas_no_interaction_2p} one can observe that the mean distance is increasing with decreasing trap frequency.  This is as expected from Fig. \ref{fig:harmonic_oscillator_potential}, where the potential is broadened with decreasing trap frequency and hence is not forcing the particles closer together. The mean distance is a bit different for brute force sampling compared to importance sampling, but the similarity might improve if more MC cycles are used.  The other value show, however,  no large difference between the results from brute force sampling and importance sampling. But to be able to compare the sampling methods more thoroughly it is better to look at the case where the system is not in the ground state. 

\begin{table}[H]\caption{Ground state energy of two electrons in harmonic oscillator trap.  Here $\overline{r}_{12}$ is the mean distance between the two electrons at positions $\bm{r}_1$ and $\bm{r}_2$ given in units of $a_0$ and $\left< E_L \right>$, $\left< T \right>$, $\left< V_{ext}\right>$  and $\left<V_{int} \right>$ are the expectation value of the local energy, kinetic energy, potential energy and interaction energy, respectively, given in units of $E_h$. The rest of the values are as explained in Tab. \ref{tab:brute_force_no_interaction_2p}. Number of MC cycles are $2^{23}$.}\label{tab:ground_state_energy_different_omegas_no_interaction_2p}
\begin{tabular}{l|l|l} 
 & Brute force: &  Importance:\\ \hline
\begin{tabular}{c} 
$\omega$  \\ \hline
1.00  \\
0.50  \\
0.10  \\
0.05  \\
0.01  \\
\end{tabular} & \begin{tabular}{ccccc}
$\alpha$ & $\left< E_L \right>$ & $\overline{r}_{12} $ & $\left< T \right>$  & $\left< V_{ext}\right>$ \\ \hline
1 & 2.00 & 1.250 & 1.0008 & 0.9992\\
1 & 1.00 & 1.775 & 0.4971 & 0.5029\\
1 & 0.20 & 3.967 & 0.0996 & 0.1004\\ 
1 & 0.10 & 5.638 & 0.0497 & 0.0503\\ 
1 & 0.02 & 12.631 & 0.0099 & 0.0101\\  
\end{tabular} & \begin{tabular}{ccccc}
$\alpha$ & $\left< E_L \right>$ & $\overline{r}_{12} $ & $\left< T \right>$  & $\left< V_{ext}\right>$ \\ \hline
1 & 2.00 & 1.254 & 0.9982 & 1.0018\\
1 & 1.00 & 1.781 & 0.4973 & 0.5027\\
1 & 0.20 & 4.046 & 0.0987 & 0.1013\\ 
1 & 0.10 & 5.534 & 0.0512 & 0.0488\\ 
1 & 0.02 & 12.488 & 0.0101 & 0.0099\\ 
\end{tabular}\\
\end{tabular}
\end{table}

\textit{Virial theorem! + energy is given by $\hbar \omega$, so energy change according to $\hbar \omega$.}

Table \ref{tab:compare_importance_alphas_2p} shows the expectation value for the energy for various parameters, $alpha$.  The calculated expectation values for the energy for the two different sampling methods are similar, especially close to the correct parameter $\alpha$, varying only by $\pm 10^{-2}$. The SEM is here underestimating the error compared to $\sigma_b$ from the blocking resampling technique for both sampling methods. However, what is different is that $\sigma_b$ is larger for importance sampling than for brute force sampling. \textit{Why is that? Is it just estimating the error more correctly (but the blocking code is the same), or is there a larger error resulting from importance sampling?}. At last, one can observe that the CPU time of the the importance sampling method is larger than brute force sampling which is expected because of the calculation of the gradient and the quantum force. But we know from Fig. \ref{fig:comparing_sampling} that the acceptance rate of brute force sampling is around 90 \% compared to importance sampling which should be close to 100 \% and this makes the importance sampling technique more effective in terms of MC cycles.

\begin{table}[H]\caption{Comparing the results for brute force sampling/importance sampling. Values are as explained in Tab. \ref{tab:brute_force_no_interaction_2p}. Number of MC cycles are 2$^{21}$.}\label{tab:compare_importance_alphas_2p}
\center
\begin{tabular}{l|l|l}
 & Brute force: & Importance:\\ \hline
\begin{tabular}{c}
$\alpha$: \\ \hline
0.50 \\
0.60 \\
0.70 \\
0.80 \\
0.90\\
1.00 \\
1.10 \\
1.20 \\
1.30\\
1.40 \\ \hline
\end{tabular} & \begin{tabular}{cccc}
 $\left< E_L \right>$: & SEM: & $\sigma_B$: & CPU time:\\ \hline
2.49074 & 0.00072 & 0.01030 & 6.26562\\
2.27645 & 0.00053 & 0.00721 & 6.53125\\
2.12481 & 0.00036 & 0.00451 & 6.59375\\
2.04873 & 0.00022 & 0.00265 & 6.68750\\
2.01135 & 0.00010 & 0.00115 & 6.82812\\
2.00000 & 0.00000 & 0.00000 & 6.75000\\
2.00863 & 0.00009 & 0.00097 & 6.59375\\
2.03343 & 0.00018 & 0.00174 & 7.29688\\
2.07103 & 0.00026 & 0.00246 & 6.75000\\
2.11725 & 0.00033 & 0.00317 & 6.70312\\ \hline
\end{tabular} & \begin{tabular}{cccc}
$\left< E_L \right>$: & SEM: & $\sigma_B$: & CPU time:\\ \hline
2.52950 & 0.00075 & 0.01952 & 8.53125\\
2.26531 & 0.00052 & 0.01281 & 8.62500\\
2.12449 & 0.00036 & 0.00813 & 8.76562\\
2.04797 & 0.00022 & 0.00454 & 9.37500\\
2.01171 & 0.00010 & 0.00203 & 8.81250\\
2.00000 & 0.00000 & 0.00000 & 8.76562\\
2.00954 & 0.00009 & 0.00171 & 8.40625\\
2.03276 & 0.00018 & 0.00305 & 8.50000\\
2.07363 & 0.00026 & 0.00440 & 8.67188\\
2.10697 & 0.00034 & 0.00536 & 8.67188\\ \hline
\end{tabular}\\
& Mean CPU time:  6.7000 & Mean CPU time:  8.7125\\
\end{tabular}
\end{table}

\subsubsection{Including optimization}

Because there are two parameters to optimize, I chose, in this project compared to the previous one, to experiment with the minimization rate during the actual optimization. I started out using a minimization rate, $\gamma$ = 0.5. It resulted in the fewest steps until the parameter value stabilized both for guesses close to the optimal value and for guesses far away from the optimal value, but for the smallest trap frequencies I had to use $\gamma = $ 0.1 or 0.2. For the case of two interacting fermions, the parameters were optimized by trying out different first guesses for $\alpha$ and $\beta$ and tuning $\gamma$ so that the parameters stabilized during the first 200 iterations. The optimal parameters were extracted from the mean of the last 50 iterations. An example run is shown in Fig. \ref{fig:example_gradient_descent} for $\omega = 0.5$.

\begin{figure}[H]
\center
\includegraphics[width=0.8\linewidth]{../Results/example_gradient_descent}\caption{Left: The development of the parameters during the steps of gradient descent. The values $\alpha_{mean}$ and $\beta_{mean}$ printed on the figure is the mean of the last 50 values. Right: The expectation value after $2^{19}$ number of MC cycles. Here also, the value printed on the figure is the mean of the last 50 iterations. }\label{fig:example_gradient_descent}
\end{figure}

For the system with six interacting fermions, the method described above was used for the largest $\omega$s (i.e. $\omega$ = 1.0, 0.5, 0.1), but for the smaller ones I had to use a smaller minimization rate (i.e. 0.01-0.05) and I also had to move step by step from $\omega$ = 0.1 to $\omega$ = 0.01 with the step size of $\Delta \omega = 0.01$. I found the parameters for the current $\omega$ and used that as a guess for the next $\omega$. I tried to do it in a more efficient way and let the simple gradient descent method find the minimum on its own, but with this unguided method, the optimization ended up in local minima at higher energies or going to infinite energies. To improve the code, I attempted to use the extended gradient descent method, described in project 1, which utilize the previous gradient to find the new parameter  \cite{project1}. But the attempt did not improve the behaviour described earlier. Compared to project 1, this systems local energy dependence on the parameters  ($\left< E_L \right> (\alpha, \beta)$), is more complicated, and also involve two parameters instead of one, which makes it harder to optimize the parameters with this simple gradient descent method.

\textit{Is the value calculated from the mean of 50 $2^{19}$ with slightly different $\alpha$s and $\beta$s, which values are oscillating around the optimal value, better than one calculation with the mean values of $\alpha$ and $\beta$ for $2^{21}$ number of MC cycles? Maybe if I also could calculate the standard deviation and also save the local energies so that I get $\sigma_b$?}

\subsubsection{Including interaction}

Here is the results of the expectation values approximating the ground state energy for two interacting electrons in a harmonic oscillator trap. The equations used to model the system is as described in the theory part of this report. That includes the trial wavefunction (see Eq. \ref{eq:trial_interacting}) with the Jastrow factor that is used to model the many-body wavefunction. Table \ref{tab:ground_state_energy_brute_force_interaction} and \ref{tab:ground_state_energy_importance_interaction} show the results of the calculations with the optimal parameters found with the gradient descent method for brute force and importance sampling, respectively. 

\textit{For two-dimensional dots, he found the energy to be E = 3 for the frequency w = 1 and E = 2/3 with w = 1/6 as the frequency.} \cite{taut1993two}

\begin{table}[H]\caption{Ground state energy of two interacting electrons in harmonic oscillator trap found with brute force sampling. Here $\overline{r}_{12}$,  $\left< T \right>$, $\left< V_{ext}\right>$  and $\left<V_{int} \right>$ are as explain in Tab. \ref{tab:ground_state_energy_different_omegas_no_interaction_2p}. The rest of the values are as explained in Tab. \ref{tab:brute_force_no_interaction_2p} . Number of MC cycles are $2^{23}$.}\label{tab:ground_state_energy_brute_force_interaction}
\center
\begin{tabular}{c|cccccrccc}
$\omega$ & $\alpha$ & $\beta$ & $\left< E_L \right>$ & SEM & $\sigma_B$ &  $\overline{r}_{12} \,\,\,$ & $\left< T \right>$  & $\left< V_{ext}\right>$ & $\left<V_{int} \right>$  \\ \hline
1.00 & 0.98847 & 0.39965 & 3.0068 & 0.00001 & 0.00009 & 1.636 & 0.8944 & 1.2990 & 0.8135\\
0.50 & 0.98061 & 0.31091 & 1.6674 & 0.00001 & 0.00010 & 2.481 & 0.4488 & 0.7051 & 0.5135\\
0.10 & 0.94693 & 0.17764 & 0.4486 & 0.00001 & 0.00011 & 6.695 & 0.1003 & 0.1767 & 0.1716\\
0.05 & 0.92747 & 0.13815 & 0.2609 & 0.00000 & 0.00011 & 10.389 & 0.0533 & 0.0997 & 0.1076\\
0.01 & 0.88398 & 0.07287 & 0.0777 & 0.00000 & 0.00006 & 29.177 & 0.0129 & 0.0284 & 0.0364\\
\end{tabular}
\end{table}

\begin{table}[H]\caption{Ground state energy of two interacting electrons in harmonic oscillator trap found with importance sampling. Here $\overline{r}_{12}$,  $\left< T \right>$, $\left< V_{ext}\right>$  and $\left<V_{int} \right>$as explain in Tab. \ref{tab:ground_state_energy_different_omegas_no_interaction_2p}. The rest of the values are as explained in Tab. \ref{tab:brute_force_no_interaction_2p}. Number of MC cycles are $2^{23}$}\label{tab:ground_state_energy_importance_interaction}
\center
\begin{tabular}{c|ccccccccc}
$\omega$ & $\alpha$ & $\beta$ & $\left< E_L \right>$ & SEM & $\sigma_B$ &  $\overline{r}_{12} \,\,\,$ & $\left< T \right>$  & $\left< V_{ext}\right>$ & $\left<V_{int} \right>$  \\ \hline
1.00 & 0.98846 & 0.39954 & 3.0069 & 0.00001 & 0.00018 & 1.643 & 0.8931 & 1.3052 & 0.8086\\
0.50& 0.98082 & 0.31068 & 1.6674 & 0.00001 & 0.00019 & 2.481 & 0.4547 & 0.6997 & 0.5130\\
0.10 & 0.94734 & 0.17810 & 0.4486 & 0.00001 & 0.00023 & 6.724 & 0.0989 & 0.1787 & 0.1710\\
0.05 & 0.92262 & 0.14090 & 0.2610 & 0.00000 & 0.00021 & 10.333 & 0.0495 & 0.1024 & 0.1091\\
0.01 & 0.88305 & 0.07366 & 0.0776 & 0.00000 & 0.00010 & 29.293 & 0.0131 & 0.0283 & 0.0362\\
\end{tabular}
\end{table}

From Tab. \ref{tab:ground_state_energy_brute_force_interaction} and \ref{tab:ground_state_energy_importance_interaction} one can observe that the two sampling methods give approximately the same results. The different expectation values for the different energies are very similar and the mean distance is also very similar. However, the standard deviation from the blocking resampling method are different. Importance sampling seems to result in larger $\sigma_B$ here as well as for the case without interaction (see Tab. \ref{tab:compare_importance_alphas_2p}). 

\textit{Have to compare with Tout's work. Say that it should be 3. Can maybe also compare that other value for another omega.}

The results above show that the energy of the system is decreasing with decreasing trap frequency, $\omega$. One can also observe that the potential energy dominates for large trap frequencies ($\omega = 1.0$ and $\omega = 0.5$), but for the smaller $\omega$s the interaction energy and the potential energy is approximately equal. For all $\omega$, the kinetic energy is the smallest one. \textit{Does this make sense? What is the potential energy? The energy stored in the force from the trap - keeping the electrons together - stronger trapping force - stronger potential energy. What is kinetic energy? Momentum? Why does it decrease with trap frequency? Because potential energy is less? Without interaction - energy oscillates between potential energy and kinetic energy - energy is conserved. }

\begin{figure}[H]
\center
\includegraphics[width=0.7\linewidth]{../Results/mean_distance_change}\caption{The mean distance between the two particles, calculated with importance sampling, compared for the situation with and without the Jastrow factor and at different trap frequencies. }\label{fig:mean_distance_compared}
\end{figure}

Figure \ref{fig:mean_distance_compared} shows the combined results from Tab. \ref{tab:ground_state_energy_different_omegas_no_interaction_2p} and \ref{tab:ground_state_energy_importance_interaction}. One can observe that the mean distance is larger for the case with interaction. This is expected since the interaction potential (see Eq. \ref{eq:hamilton_interaction}) is a repulsive potential, and will hence force the electrons further apart. One can also observe that the ratio of the two different cases increase with decreasing trap frequency, i.e. the mean distance increases more for smaller trap frequencies. \textit{is that as expected? Why? Does that mean that the trap frequency is more important than the interaction?}

\subsubsection{One-body density}

The one-body density can tell how the particles are distributed in space, and Fig. \ref{fig:one_body_density_no_interaction_2p} shows the result of the one-body density of the system of two not interacting fermions for different trap frequencies. One can observe, as one saw in the development of the mean distance (see Fig. \ref{fig:mean_distance_compared}), that the particle distribution or density is spread out in space for smaller trap frequencies. From Fig. \ref{fig:one_body_density_no_interaction_2p} one can also observe that the electrons spend most of their time around the origin of the harmonic oscillator trap. The shape of the curve seems not to change for the different trap frequencies, only the height and width.

\begin{figure}[H]
\center
\includegraphics[width=0.85\linewidth]{../Results/one_body_density_no_interaction_2p}\caption{One-body density of the two electron system without the Jastrow factor, but with different trap frequencies. Left: The larger trap frequencies. Right: The smaller trap frequencies. The system with $\omega = 0.1$ is in both plots to show the difference in scale for the two plots. Used $2^{24}$ MC cycles. }\label{fig:one_body_density_no_interaction_2p}
\end{figure}

To get a smooth curve, I found out that you either have to use larger bin sizes or more MC cycles. Since the one-body density is found by counting how many times a particle is situated within a certain bin, i.e. with a certain distance to origin, more samples of the positions will give a more true distribution. So if the bin size is small, you need more MC cycles to smooth out the random difference between the two bins. On the other hand, if the bin size is too large real differences or fast changes in the distribution might not be detected.

\begin{figure}[H]
\center
\includegraphics[width=0.85\linewidth]{../Results/one_body_density_no_interaction_2p_few_points}\caption{One-body density of the two electron system without the Jastrow factor, but with different trap frequencies. Left: The larger trap frequencies. Right: The smaller trap frequencies. The system with $\omega = 0.1$ is in both plots to show the difference in scale for the two plots. Here showed for a case with larger bins then Fig. \ref{fig:one_body_density_no_interaction_2p}. Used $2^{24}$ MC cycles.}\label{fig:one_body_density_no_interaction_2p_fewer_bins}
\end{figure}

Furthermore, Fig. \ref{fig:one_body_density_interaction_2p} shows the one-body densities with and without the Jastrow factor. Here, one can also observe the same as was seen with the mean distance, that the Jastrow factor spreads out the distribution of the particles, but the difference is a lot clearer in the one-body density plot. Another interesting observation is that, for the interacting case, the maximum does not seem to lay at the origin. It is not that easy to see for the larger trap frequencies, but one can easily see it for $\omega = 0.1$ in the right part of Fig. \ref{fig:one_body_density_interaction_2p}. This seems to be reasonable because there is a repulsive force between the particles, they spend more time further away from the origin. \textit{Could it be that for larger trap frequencies there is a larger energy-wise gain ( i.e. the energy is lower) to be situated closer to origin, hence the maximum is not shifted as far from origin for those frequencies, but for the lower trap frequencies the gain obtained closer to origin does not compensate enough of the repulsive forces (or interaction energy), so the maximum is shifted away from origin. That should imply that the position of the maximum reflects a balance between the potential energy and the interaction energy. }

\begin{figure}[H]
\center
\includegraphics[width=0.85\linewidth]{../Results/one_body_density_interaction_2p}\caption{One-body density of the two electron system with and without the Jastrow factor and with different trap frequencies (indicated by the color of the lines). Dashed lines are with the Jastrow factor and filled lines are without Jastrow factor. Left: The larger trap frequencies. Right: The smaller trap frequencies.  The system with $\omega = 0.1$ is in both plots to show the difference in scale for the two plots. Used $2^{24}$ MC cycles. }\label{fig:one_body_density_interaction_2p}
\end{figure}

\subsection{Extending to more particles}

To investigate the quantum dot system further, the next shell (illustrated in Fig. \ref{fig:states}) is filled which results in a system of six particles. Furthermore, the third shell is filled resulting in a system of twelve particles. 

\subsection{Six particles}

The expectation value for the total energy (i.e. the local energy), the kinetic energy and the potential energy for the different trap frequencies and without interaction are listed in Tab. \ref{tab:ground_state_energy_importance_6p}. One can observe that for all the trap frequencies, the kinetic and potential energy seems to follow the virial theorem (see Eq. \ref{eq:virial_theorem}), if one allows for a small difference of maximum $\pm 0.03$ a.u.. 

\begin{table}[H]\caption{Ground state energy of six electrons in harmonic oscillator trap. Here $\left< E_L \right>$, $\left< T \right>$ and $\left< V_{ext}\right>$  are as explain in Tab. \ref{tab:ground_state_energy_different_omegas_no_interaction_2p}. Number of MC cycles are $2^{23}$. }\label{tab:ground_state_energy_importance_6p}
\center
\begin{tabular}{c|rcc}
$\omega$ & $\left< E_L \right>$  & $\left< T \right>$  & $\left< V_{ext}\right>$ \\ \hline
1.00 & 10.00 & 4.9865 & 5.0135\\ 
0.50 & 5.00 & 2.4973 & 2.5027\\
0.10 & 1.00 & 0.4861 & 0.5139\\
0.05 & 0.50 & 0.2483 & 0.2517\\
0.01 & 0.10 & 0.0454 & 0.0546\\
\end{tabular}
\end{table}

Furthermore, the result after including interaction is showed in Tab. \ref{tab:ground_state_energy_importance_interaction_6p}. One can observe, if one compares the result with Tab. \ref{tab:ground_state_energy_importance_interaction}, that the both the SEM and $\sigma_b$ are larger for the system with six electrons. This could be explained by the fact that the total energies are larger for this system, and hence the errors will be larger. If one compare the error in as a fraction for the case with $\omega = 1.0$ $\nicefrac{\sigma_b}{\left< E_L \right>} = 5.99 \cdot 10^{-5}$ for two electrons and $3.98\cdot 10^{-4}$ for six electrons and one can see that the error is larger for the six electron case as a fraction as well. \textit{Would the blocking error be worse if the optimal parameters where worse? I guess not because it sort of says how good the expectation value is for the parameters given. It tells if the energy that comes out can be relied upon. It does not tell us how far we are from the real energy.}

\begin{table}[H]\caption{Ground state energy of six interacting electrons in harmonic oscillator trap found with importance sampling. Number of MC cycles are $2^{23}$.}\label{tab:ground_state_energy_importance_interaction_6p}
\center
\begin{tabular}{c|cccccccc}
$\omega$ & $\alpha$ & $\beta$ & $\left< E_L \right>$ & SEM & $\sigma_B$ &  $\left< T \right>$  & $\left< V_{ext}\right>$ & $\left<V_{int} \right>$  \\ \hline
1.00 & 0.71567 & 0.49372 & 20.4492 & 0.00022 & 0.00813 & 2.3429 & 10.7076 & 7.3988\\
0.50 & 0.75823 & 0.34260 & 11.9868 & 0.00011 & 0.00522 & 1.3226 & 5.8094 & 4.8548\\
0.10 & 0.78852 & 0.15041 & 3.6542 & 0.00003 & 0.00416 & 0.2951 & 1.7035 & 1.6556\\
0.05 & 0.76518 & 0.10733 & 2.2223 & 0.00003 & 0.00436 &  0.1178 & 1.0882 & 1.0162\\
0.01 & 0.64907 & 0.05085 & 0.7191 & 0.00001 & 0.00586 & 0.0021 & 0.3803 & 0.3367\\
\end{tabular}
\end{table}

\textit{Compare with Even's values?}

I wanted to take a closer look at the optimized parameters for the two different systems too, especially since I had so much trouble finding the parameters for the six electron system. Figure \ref{fig:parameters_compared} compares the optimized parameters for the two systems with different amounts of particles at the different trap frequencies. One can observe that the $\beta$ parameter which is involved in the Jastrow factor is similar for both systems and also show a similar development from one trap frequency to the next. This is expected, I guess, since the interaction between the particles do not change that much by increasing the number of particles. The other parameter $\alpha$, however, which is involved in the Slater determinant, shows completely different values and development for the two systems. This is also reasonable because the wave function, i.e. the Slater determinant part, changes a lot when we increase the number of particles. Because when the number of particles is increased higher states are filled and the Slater determinant is changed a lot. The development of $\alpha$ though is strange, I would have expected it to either increase or decrease with an increasing trap frequency, but In the right part of Fig. \ref{fig:parameters_compared} one can observe that the largest $\alpha$ is found for $\omega = 0.1$ which is the middle $\omega$ investigated. But my expectation might be wrong, maybe the strange development is due to the single-particle states that are introduced with the higher energy states. 

\begin{figure}[H]
\center
\includegraphics[width=0.85\linewidth]{../Results/parameters_compared}\caption{Left: The optimized parameters for the system with two particles and with the different trap frequencies. Right: The optimized parameters for  for the system with six particles and with the different trap frequencies. All found with the simple gradient descent method.}\label{fig:parameters_compared}
\end{figure}

Next, Fig. \ref{fig:energy_per_particle_compared} compares the energy per particle of the two systems when interaction is included. The left part of the figure shows that the magnitude of kinetic energy per particle and the development of the kinetic energy with trap frequency is very similar for both systems (two and six particles). The right part of the figure shows that both the potential and interaction energy per particle is larger for the six particle system at the same trap frequency, than the two particle system. This is as expected since the trap frequency determine the "size" of the trap. The six particle system's potential energy is hence larger for the same size system when there are more particles involved. The interaction energy is also larger when more particles, that are repulsed by each other, are forced together in the "same amount of space" (it is not exactly the same amount of space since the space is not strictly limited, but it is intuitive to image it like that). Apart from this the six particle system show the same behaviour as the two particle system, energy wise. 

\begin{figure}[H]
\center
\includegraphics[width=0.85\linewidth]{../Results/energy_per_particle_compared}\caption{Left: The expectation value for the kinetic energy per particle for the two different systems examined at the different trap frequencies. Right: The expectation value of the potential and interaction energy per particle for the two different systems at the different trap frequencies.  }\label{fig:energy_per_particle_compared}
\end{figure}



\begin{figure}[H]
\center
\includegraphics[width=0.85\linewidth]{../Results/one_body_density_no_interaction_6p}\caption{One-body density of the six electron system without the Jastrow factor, but with different trap frequencies. Left: The larger trap frequencies. Right: The smaller trap frequencies. The system with $\omega = 0.1$ is in both plots to show the difference in scale for the two plots. Used $2^{24}$ MC cycles.}\label{fig:one_body_density_no_interaction_6p}
\end{figure}

\begin{figure}[H]
\center
\includegraphics[width=0.85\linewidth]{../Results/one_body_density_interaction_6p}\caption{One-body density of the six electron system with and without the Jastrow factor and with different trap frequencies (indicated by the color of the lines). Dashed lines are with the Jastrow factor and filled lines are without Jastrow factor. Left: The larger trap frequencies. Right: The smaller trap frequencies.  The system with $\omega = 0.1$ is in both plots to show the difference in scale for the two plots. Used $2^{24}$ MC cycles. }\label{fig:one_body_density_interaction_6p}
\end{figure}

\textit{Strange dump in the middle for $\omega = 0.1$ with Jastrow factor. Have the larger trap opened up so that the particles can rearrange themselves with particles spending more time in the middle of the trap. }

\subsection{Twelve particles}

\begin{table}[H]\caption{Number of MC cycles are $2^{23}$. }\label{tab:ground_state_energy_importance_12p}
\center
\begin{tabular}{c|rrr}
$\omega$ & $\left< E_L \right>$  & $\left< T \right>\,\,\,$  & $\left< V_{ext}\right>\,$ \\ \hline
1.00 & 28.00 & 14.0117 & 13.9883\\ 
0.50 & 14.00 & 7.0463 & 6.9537\\ 
0.10 & 2.80 & 1.4084 & 1.3916\\ 
0.05 & 1.40 & 0.6901 & 0.7099\\ 
0.01 & 0.28 & 0.1419 & 0.1381\\ 
\end{tabular}
\end{table}

\begin{table}[H]\caption{Ground state energy of twelve interacting electrons in harmonic oscillator trap found with importance sampling. Number of MC cycles are $2^{23}$}\label{tab:ground_state_energy_importance_interaction_12p}
\center
\begin{tabular}{c|cccccccc}
$\omega$ & $\alpha$ & $\beta$ & $\left< E_L \right>$ & SEM & $\sigma_B$ &  $\left< T \right>$  & $\left< V_{ext}\right>$ & $\left<V_{int} \right>$  \\ \hline
\end{tabular}
\end{table}


\begin{figure}[H]
\center
\includegraphics[width=0.85\linewidth]{../Results/one_body_density_no_interaction_12p}\caption{One-body density of the six electron system without the Jastrow factor, but with different trap frequencies. Left: The larger trap frequencies. Right: The smaller trap frequencies. The system with $\omega = 0.1$ is in both plots to show the difference in scale for the two plots. Used $2^{24}$ MC cycles. }\label{fig:one_body_density_no_interaction_12p}
\end{figure}

\subsection{Efficiency}

Show results or errors of attempt to make parts of code more efficient. 

\subsection{Performance analysis}

\subsubsection{Vectorization}

\subsubsection{Parallellization}

Chose to parallellize using cmake.