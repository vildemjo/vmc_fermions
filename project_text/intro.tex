The main objective of this project is to investigate the many-body problem of simulating a quantum dot. A quantum dot is basically electrons that are trapped in an electrical potential. In this project, the potential is modelled as a harmonic oscillator potential. This assumption is ... (maybe not accurate because ...), but it is an easy potential to start with and the potential can later be expanded and most of the code can be reused as it is. Furthermore, we use a trial wavefunction with one (two) fitting parameters for the case without (with) a Jastrow factor to represent the interaction between the particles. This trial wave function is not the exact wave function, hence the simulation can only approximate the exact values for the ground state energy.

The report starts out by introducing the system with the representative equations and analysis tools. Most of the numerical tools used in the programming in this project has already been described in detail in project 1 \cite{project1}, but some additional things are explained or more thoroughly elaborated on in the theory and method part of this report.

Furthermore, the results and discussion part first analyses how different important parameters have been chosen, i.e. choise of evaluation of the double derivative, step size of sampling techniques and method and parameters of the optimization. In addition the expectation energies and the one-body densities of the systems involves are compared and evaluated. At last, an evaluation of code's efficiently is made and some concluding remarks are stated.