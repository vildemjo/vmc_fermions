The main objective of this project is to investigate the many-body problem of simulating a quantum dot. A quantum dot is basically electrons that are trapped in an electrical potential. In this project, the potential is modelled as a harmonic oscillator potential. In this project Variational Monte Carlo (VMC) methods are used to solve the many-body SE for electrons. This project is about the so-called full-shell systems of two, six and twelve electrons in harmonic oscillator traps of different strengths, i.e. trap frequencies. The systems is investigated both for the case where interaction is neglected and included.

The report starts out by introducing the system with the representative equations and analysis tools. Most of the numerical tools used in the programming in this project has already been described in project 1 \cite{project1}, but some additional things are explained or more thoroughly elaborated on in the theory and method part of this report. Furthermore, the results and discussion part first analyses how different important parameters have been chosen, i.e. choice of evaluation of the double derivative, step size of sampling techniques and method and parameters of the optimization. In addition the expectation energies and the one-body densities of the systems involves are compared and evaluated. At last, an evaluation of code's efficiently is made and some concluding remarks are stated.